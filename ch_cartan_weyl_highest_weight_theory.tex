\chapter{Cartan-Weyl Highest Weight Theory}
\label{cha:cartan_weyl_highest_weight_theory}
References: \cite[$\S 20$]{humphreys1972introduction}

In this chapter $\LieG$ denotes a semisimple Lie algebra over an algebraically closed field $\FF$ of characteristic $0$, $\LieH$ is a fixed Cartan subalgebra of $\LieG$, $\Phi$ the root system $\Delta = \{\alpha_1,\dotsc,\alpha_l\}$ a base of $\Phi$, $\Weyl$ the Weyl group.

\section{Weights and maximal vectors}
\label{sec:weights_and_maximal_vectors}

If $V$ is a finite dimensional $\LieG$-module, it follows from \Todo{Theorem~\ref{thm:preservation_of_jordon_decomposition} (preservation of Jordon decomposition)}  that $\LieH$ acts diagonally on $V$. That is 
\begin{align}
    V &= \coprod_{\lambda\in \LieH^* } V_\lambda, \quad \text{with}\\
    V_\lambda &= \{v\in V\ |\ h\cdot v = \lambda(h) v \ \forall h\in H\}
\end{align}
$V_\lambda$ is called the \defn{weight space} if $V_\lambda\neq 0$ and $\lambda$ is called a \defn{weight} of $H$ on $V$. 
These definitions make sense even when $V$ is infinite dimensional.
\begin{insight}
    Roots are just the weights of the adjoint representation, with $\LieG_\alpha$  as the weight space (dimension one).
\end{insight}

\begin{lemma}
    Let $V$ be any arbitrary $\LieG$-module. Then
    \begin{enumerate}[(a)]
        \makethislistcompact
        \item 
            $\LieG_\alpha$ maps $V_\lambda$ to $V_{\lambda+\alpha}$ for $\lambda\in \LieH^*,\,\alpha\in\Phi$
        \item 
            If $V'=\sum_{\lambda\in \LieH^*} V_\lambda$ is direct, and $V'$ is a $\LieG$-submodule of $V$.
        \item
            If $\dim V < \infty$, then $V'=V$
    \end{enumerate}
\end{lemma}
\begin{proof}
    For $x\in \LieG_\alpha$ and $v\in V_\lambda$  we have
    \begin{align*}
        h(xv) &= [h,x]v - x(hv)\\
            &= \alpha(h)xv - x\lambda(h)v \\
            &= (\alpha(h) + \lambda(h)) xv
    \end{align*}
\end{proof}

A \defn{maximal vector} (of weight $\lambda$) in a $\LieG$-module $V$ is a non-zero vector $v^+\in V_\lambda$ annihilated by all $\LieG_\alpha\ (\alpha > 0)$. 

\begin{insight}
    If $\LieG$ is simple Lie algebra and $\beta$ is a maximal root in $\Phi$ relative to $\Delta$, then any nonzero element of $\LieG_\beta$ is a maximal vector for the adjoint representation of $L$.
    If $\dim V = \infty$, then there may not be a maximal vector.
\end{insight}


A \defn{Borel subalgebra} is
\begin{align}
    B(\Delta) &= \LieH\oplus \bigoplus_{\alpha>0} \LieG_\alpha.
\end{align}
\begin{lemma}
    If $\dim V < \infty$ , then a maximal vector exists.
\end{lemma}
\begin{proof}
    \Todo{$B(\Delta)$ is a solvable subalgebra. By Lie's theorem, a solvable subalgebra contains a common eigenvector for all endomorphisms in $\LieG$.} 
\end{proof}

%By Lie's theorem, $B(\Delta)$ has a common eigenvector (killed by all \gamma_\alpha,)

A \defn{standard cyclic module} (weight $\lambda$) is a $\LieG$-module $V$ such that 
\begin{align*}
    V=U(\LieG)\cdot v^+
\end{align*}
and we call $\lambda$ the \defn{highest weight} of V. 
\begin{insight}
    The term \emph{cyclic} is often used to refer to something that is generated by one element.
\end{insight}

\begin{theorem} 
    Let $V$ be a standard cyclic $\LieG$-module, with maximal vector $v^+\in V_\lambda$. Let $\Phi^+=\{\beta_1,\dotsc,\beta_m\}$. Then
    \begin{enumerate}[(a)]
        \makethislistcompact
        \item $V$ is spanned by the vectors $y_{\beta_1}^{i_1}\cdots y_{\beta_m}^{i_m}\cdot v^+\ (i_j\in \ZZ^+)$; in particular, $V$ is the direct sum of its weight spaces.
        \item  The weights of $V$ are of the form $\mu = \lambda - \sum_{i=1}^{l}k_i \alpha_i$,  that is, all weights satisfy $\mu < \lambda$.
        \item For each $\mu \in \LieH^*$, $\dim V_\mu < \infty$ and $\dim V_\lambda = 1$.
        \item Each submodule of $V$ is the direct sum of weight spaces.
        \item V is an indecomposable $\LieG$-module, with a unique maximal (proper) submodule and a corresponding unique irreducible quotient.
        \item Every nonzero homomorphic image of $V$ is also standard cyclic of weight $\lambda$.
    \end{enumerate}
\end{theorem}

\section{Existence and uniqueness of highest weight representations}
\label{sec:existence_and_uniqueness_of_highest_weight_representations}
Reference: \cite[$\S 20.3$]{humphreys1972introduction}

\begin{theorem}
    Irreducible standard cyclic modules are isomorphic.
\end{theorem}
This theorem proves the uniqueness of highest weight representations. The next part is to prove the existence. We do this by construction.

The first type of construction is called \defn{Verma modules.}


