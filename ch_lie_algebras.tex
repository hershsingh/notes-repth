\chapter{Lie Algebras}
\label{cha:lie_algebras}
In this chapter, we undertake a systematic study of Lie algebras and attempt to classify them.

\section{Definitions}
%\label{sec:definitions}

Let's make a few definitions.
\begin{align}
    D\LieG &= [\LieG, \LieG] &\text{}\\
    D_k\LieG &= [\LieG, D_{k-1}\LieG] &\text{lower central series}\\
    D^k\LieG &= [D^{k-1}\LieG, D^{k-1}\LieG] &\text{derived series}
\end{align}

\begin{definition} A lie algebra $\LieG$ is 
    \begin{enumerate}[(i)]
        \makethislistcompact
        \item \textbf{nilpotent} if $D_k\LieG=0$ for some $k$,
        \item \textbf{solvable} if $D^k\LieG=0$ for some $k$, and
        \item \textbf{semi-simple} if it has no non nonzero solvable ideals.
    \end{enumerate}
\end{definition}

\begin{definition}[Radical]
    Sum of all solvable ideals in $\LieG$ is again a solvable ideal, called the \textbf{radical} of $\LieG$ and denoted by $\Rad(\LieG)$. This is the maximal solvable ideal.
\end{definition}

\begin{lemma}[Equivalent definitions of a semisimple Lie algebra]
    \label{lemma:semisimple_definitions}
    A Lie algebra is semisimple iff 
    \begin{enumerate}[(i)]
        \makethislistcompact
        \item it has no non zero solvable ideals;
        \item $\Rad \LieG=0$;
        \item it has no non-zero abelian ideals.
    \end{enumerate}
\end{lemma}
\begin{proof}
    The first two are obvious. We will prove that they are equivalent to the third.
    Let $\LieG$ be a semisimple Lie algebra. It does not contain any non zero abelian ideal since abelian ideals are solvable. On the other hand, if it is not semisimple, then let $\Lie r = \Rad \Lie g \neq 0$. Since $\Lie r$ is solvable, $D^k\Lie r = [D^{k-1}\Lie r, D^{k-1}\Lie r]=0$ for some minimum $k$. $D^{k-1}\Lie r$ is thus an abelian ideal.

\end{proof}

\begin{definition}[Reductive]
    A lie algebra $\LieG$ is \textbf{reductive} if it is the direct sum of a semisimple lie algebra and an abelian lie algebra.
\end{definition}
\begin{insight}
   It does not make much sense to classify Lie algebras and keep $\Lie{gl}_n$ out of it. That's why we define reductive Lie algebras. Even though $\Lie{gl}_n$ is not semi-simple, it is reductive as $\Lie{gl}_n = \Lie{sl}_n\oplus \Lie{t}$ is the direct sum of the traceless Lie algebra with with the scalar Lie algebra.
\end{insight}

\begin{lemma}
    $ (\mathfrak{a} + \mathfrak{b})/\mathfrak{b} \isomorphic \mathfrak a / (\mathfrak a \cap \mathfrak b) $
\end{lemma}
\begin{proof} The kernel of the composite map 
   \begin{align}
       \LieA + \LieB \onto \LieA \onto \LieA/(\LieA \cap \LieB)
   \end{align} 
   is the ideal $\LieB \in \LieA + \LieB$.
\end{proof}


Notice that the lie algebra $\LieG/\Rad(\LieG)$ is semisimple. Any lie algebra fits into the exact sequence
\begin{align}
    0\to \Rad(\LieG) \to \LieG \to \LieG/\Rad(\LieG) \to 0
\end{align}
where the first algebra is solvable and the last algebra is semisimple. Our approach to classify representations of Lie  algebras is then to study the representations of solvable and semisimple Lie algebras.



\section{Overview}
%\label{sec:overview}




\begin{theorem}[Ado's theorem]
Every finite dimensional lie algebra is a subalgebra of $\LieGL(V)$.
\end{theorem}

\begin{theorem}[Lie's theorem]
    
\end{theorem}

\vspace{1em}
\hrule
\vspace{1em}

Reference: FH 8.3
\begin{proposition}
    Let $G$ be a Lie group and $\LieG$ be the its Lie algebra. Let $\LieH\in \LieG$ be a subalgebra. 
    Then the subgroup of $G$ generated by $\exp(\LieH)$ is an immersed subgroup $H$ with tangent space $T_eH=\LieH$
\end{proposition}
\begin{proof}
    \Todo{complete this.} 
\end{proof}


\begin{insight}
    Let $\{(e^{\iota \alpha t}, e^{\iota\beta t})| t\in \RR\} \subseteq S^2 $. This is dense in $S^2$ if $\alpha/\beta$ is irrational. So, it's not closed. It's important to keep in mind the difference in closed and immersed groups. 
\end{insight}



\section{Covering Space}
\label{sec:covering_space}

\begin{figure}[h]
    \centering
    \begin{tikzpicture}[disc/.style={top color=white, bottom color=white!80!black}]
        \def\dist{0.2}
        \def\xrad{1}
        \def\yrad{0.3}
        \draw [disc] (0,-13*\dist) ellipse [start angle=0, end angle=360, x radius=\xrad, y radius=\yrad] node (U) {};
        \draw [-stealth, thick] (0,-7*\dist) -- (0,-11*\dist) node [midway, right] {$\pi$};
        \draw [disc] (0,-5*\dist) ellipse [start angle=0, end angle=360, x radius=\xrad, y radius=\yrad] node (bottom) {};
        \draw [dotted, very thick] (0,-5*\dist) -- (0,-\dist);
        \draw [disc] (0,-\dist) ellipse [start angle=0, end angle=360, x radius=\xrad, y radius=\yrad];
        \draw [disc] ellipse [start angle=0, end angle=360, x radius=\xrad, y radius=\yrad] node (top) {};

        \node [left of=U, node distance=1.7cm] {$U$};

        %\begin{scope}[shift={(-1.5cm,0cm)}]
        %\draw [decorate, decoration={brace, mirror, amplitude=2pt}] node [left of=top, node distance=1.5cm] {} -- node [left of=bottom] {};
        \def\xx{1.2cm}
        \draw [decorate, decoration={brace, mirror, amplitude=4pt, raise=2pt}, thick] ($(top)- (\xx,0)$) -- ($(bottom) - (\xx,0)$) node [midway, left, xshift=-4pt] {$\pi^{-1}(U)$};
        %\end{scope}

    \end{tikzpicture}
    \caption{Covering Space}
    \label{}
\end{figure}

A covering space is a space $C$ with a continuous \emph{surjective} map 
\begin{align}
    \pi: C\to X
\end{align}
such that for any point $x\in X$, there exists an open neighbourhood $U$ of $x$ such that 
\begin{align}
    \pi^{-1}(U)   &=  \coprod U_\alpha \quad\text{and}\\
    \pi|_{U_\alpha}&: U_\alpha \xrightarrow{\isomorphic} U
\end{align}
where the $U_\alpha$ are open sets and $\coprod$ is the disjoint union.
\begin{insight}
    Example 
    \begin{align}
        \RR\onto \RR/\ZZ
    \end{align}
    Every quotient is not a cover.
\end{insight}

\subsection{Universal Cover}
\label{sub:universal_cover}


A covering space is a \emph{universal covering space} if it is simply connected. 

It is universal in the sense that if there is any other cover $C$ of $X$, then there is a covering map $f: D\to C$ such that following diagram commutes
\begin{center}
\begin{tikzpicture}[commutative diagram]
    \node [] (X) {$X$};
    \node [above left of=X] (D) {$D$};
    \node [above right of=X](C) {$C$};

    \draw [arrow] (D) -- (X) node [midway, auto=right] {$q$};
    \draw [arrow, dashed] (D) -- (C) node [midway, above] {$f$};
    \draw [arrow] (C) -- (X) node [midway, auto=left] {$\pi$};

    \draw [->, opacity=0.3] (0,1) arc (90:-250:5pt);

    
\end{tikzpicture}
\end{center}

\begin{insight}
   The universal cover of the space $X$ covers all the connected covers of the space $X$.
\end{insight}

\subsection{Second Principle}
\label{sub:second_principle}

Let $G$ and $H$ be Lie groups, $G$ is simply connected. Let $\LieG, \LieH$ be the associated Lie algebras. A linear map $\alpha: \LieG \to \LieH$ is the differential of a map $A: G\to H$ of Lie groups if and only if $\alpha$ is a map of Lie algebras.

\begin{proposition}
   A linear map $\alpha: \LieH \to \LieG$  is an isomorphism iff $A: H\to G$ is an isogeny.
\end{proposition}

\begin{proof}  {[\Todo {Sanitize; Lecture 10.17.2014}]} \\
Isogeny implies that there exists a $U\ni I_h$, such that
\begin{align}
    \varphi|_U : U_h \xrightarrow{\ \isomorphic\ } \varphi(U_h)
\end{align}
is an isomorphism.
Suppose, conversely, that $d\varphi: \LieH\to \LieG$ is an isomorphism. Then there exists a neighbourhood $U\ni I_h$ such that $    \varphi: U\to \varphi(U)$ is an isomorphism.

\begin{align}
    \varphi^{-1} (I_g) &= \bigcup_\alpha h_\alpha 
\end{align}
If $h_\alpha \neq I$, then $h_\alpha \not\in U$. Therefore, $\bigcup h_\alpha$ is a disjoint union.


Then $\coprod_\alpha U h_\alpha h\to \varphi(U)g$  is the cover of $U_g$ whenever $\varphi(h)=g$.

\end{proof}

\subsection{A bit on isogeny}
\label{sub:a_bit_on_isogeny}

Let $G$ be a group, $H$ a connected manifold. Let $\varphi: H\to G$ be a covering space map. Let $e' \mapsto e_G$. Then there exists a unique Lie group structure on $H$ such that  $e' = \id_H$ and $\varphi: H\to G$ is a Lie group morphism.

\begin{proposition}
   Let   
\end{proposition}

\begin{proof}[Proof of Second Pricniple]
    \Todo{Complete this}
\end{proof}

\section{Rough Classification of Lie Algebras}
\label{sec:rough_classification_of_lie_algebras}

We have three major theorems in this section:
\begin{enumerate}
    \makethislistcompact
    \item Engel's theorem,
    \item Lie's theorem,
    \item Levi decomposition theorem,
    \item Ado's theorem
\end{enumerate}

\subsection{Levi decomposition}
\label{sub:levi_decomposition}

We have the short exact sequence
\begin{align}
    \Rad \LieG \hookrightarrow \LieG \onto \LieG/\Rad\LieG
\end{align}
Levi's theorem basically states that this sequence splits, that is, there is an subspace $\Lie{l}$ such that
\begin{align}
    \LieG = \Lie{l}\oplus\Rad\LieG
\end{align}

\hrule\vspace{1em}
\Todo{Sanitize; Lecture 10.17.2014}

\begin{theorem}[Levi's theorem]
   Let $\LieG$  be a lie algebra with the radical $\Lie r$. There exists a subalgebra $\Lie l$ such that $\Lie g = \Lie l \oplus \Lie r$.
\end{theorem}
\begin{insight}
    This looks very much like 
    \begin{align}
        \begin{pmatrix}  x& y \\ 0 & z \end{pmatrix} &= \begin{pmatrix} x & 0 \\ 0 & z \end{pmatrix} + \begin{pmatrix} 0 & y \\ 0 & 0  \end{pmatrix}
    \end{align} 
\end{insight}
\begin{proof}
    Reduction step: No non-zero ideal is properly contained in $\Lie r$. (Otherwise if $\Lie i\subset \Lie r$ then $\Lie r/\Lie i \in \Lie g/ \Lie l$) 
     In particular, we can assume that $\Lie r $ is abelian (otherwise $0\neq D\Lie r\subset \Lie r$).

     $D\Lie r$ is preserved by all derivations of $\LieG$ (that is, all linear maps $LieG\to LieG$ satisfying the Leibniz rule). In particular, $D\Lie r$ is a radical of $\LieG$.
     So, $\Lie r$ is abelian. We may also assume that $[\Lie g, \Lie r]\neq 0$ (otherwise, )
    
    \Todo{....Complete this}.

    
\end{proof}

\subsection{Engel's theorem}
\label{sub:engel_s_theorem}

\begin{theorem}[Engel's theorem]
    Let $\LieG\in \LieGL(V)$ be any Lie subalgebra such that every $X\in \LieG$ is a nilpotent endomorphism of $V$. Then there exists a nonzero vector $v\in V$ such that $X(v)=0$ for all $X\in\LieG$.
\end{theorem}
\begin{corollary}
    
\end{corollary}
    This implies that there exist a basis in which $\LieG$ is upper triangular.

\subsection{Lie's Theorem}
\label{sub:lie_s_theorem}
\begin{theorem}[Lie's Theorem]
    Let  $\LieG\in\LieGL(V)$ be a complex solvable Lie algebra. Then there exists a nonzero vector $v\in V$ that is an eigenvector of $X$ for all $X\in \LieG$.
\end{theorem}

\section{Complete Reducibility aka Semisimplicity}
\label{sec:complete_reducibility_aka_semisimplicity}

\Todo{Lecture 10.22.2014}

\begin{definition}[Semisimplicity]
    A representation of a group or a Lie algebra is \emph{semisimple} if given a subrepresentation $W\in V$, there exists another invariant subspace $W'\in V$ such that $W\cup W'=\varnothing$ and $V=W\oplus W'$. 

In such a case, $W'$ is said to be complement to $W$.
\end{definition} 


\begin{insight}
    As an example of a Lie algebra which is not semisimple, consider the Borel algebra defined as
    \begin{align}
        B= \left\{ \begin{pmatrix} x & y \\ 0 & z \end{pmatrix} \in \GL_2(\RR)\right\} .
    \end{align}
    The vector $ \begin{pmatrix} 1 \\ 0 \end{pmatrix}$ spans an invariant subspace but has no complement.
\end{insight}

\subsection{Killing Form}
\label{sub:killing_form}

\begin{itemize}
    \makethislistcompact
    \item \Todo{Lecture 10.22.2014}
    \item \cite{humphreys1972introduction} Section II.4
    \item \cite{fulton_representation_1991} Section 14.2
\end{itemize}

\Todo{Lecture on 10.24.2014}

\begin{proposition}
    A Lie algebra is semisimple iff the Killing form is degenerate.
\end{proposition}
\begin{proof}
    \Todo{Complete this}
\end{proof}

We have two notions of semisimplicity. One is the direct sum of simpler objects, and the other one is \Todo{...}. The following corollary talks about about the first notion.
\begin{corollary}
   A semisimple Lie algebra is a direct product of simple Lie algebras. (This implies that $\LieG=D\LieG$ if $\LieG$ is a semisimple Lie algebra.)
\end{corollary}
\begin{proof}
    For any ideal $\LieH \subset \LieG$, 
    \begin{align}
        \LieH^\dagger = \{x\in \LieG\ |\  B(x,\LieH)=0 \}
    \end{align}
    is an ideal, since $ B([x,y],\LieH) = B(x,[y,\LieH]) \subset B(x,\LieH) = 0$. By Cartan's criterion, $\LieH^\dagger\cap \LieH$ is solvable, but $\LieG$ is semisimple, so $\LieH^\dagger\cap\LieH=0$. \Todo{Complete this.}
\end{proof}


\subsection{Complete Reducibility}
\label{sub:complete_reducibility}
A $\LieG$-module $V$ is called \emph{completely reducible} if $V$ is a direct sum of irreducible $\LieG$-submodules, or equivalently, if each $\LieG$-submodule $W\subset V$ has a complement $W'\in W$.




\begin{theorem}[Weyl's Theorem]
    \label{thm:weyl}
    Finite dimensional reprsentations of a semisimple Lie algebra are completely reducible. 

    Equivalently, let $V$ be a representation of a semisimple Lie algebra $\LieG$. Let $W\subset V$ be a submodule. There exists a complement $W'$, that is, a subrepresentation $W'\subset V$ such that $W+W'=V$ and $W\cap W'=0$.
\end{theorem}
\begin{proof}
   Introduce the Casimir operator. \Todo{Complete this.}
\end{proof}

\begin{insight}
   %Another approach to complete reducibility 
   Representations of $\SL_n(\CC)$ are the same as representations of $\SU_n(\RR)$. Notice that even though $\SU_n(\RR) \injects \SL_n(\CC)$, it is \emph{not} a complex Lie group. It is a real Lie group.
   \Todo{More to come...}
\end{insight}

\subsection{Jordon-Chevalley Decomposition}
\label{sub:jordon_chevalley_decomposition}

\emph{Reference: \cite{humphreys1972introduction} Section 4.2}

Here $\FF$ is a field of arbitrary characteristic.

An element $x\in \End(V)$ (for $V$ finite dimensional) is called \emph{semisimple} if the roots of its minimal polynomial over $\FF$ are all distinct. For $\FF$ an algebraically closed field, \Todo{this is equivalent to the statement that $x$ is semisimple iff $x$ is diagonalizable.}

\begin{insight}
   A Lie alegebra consisting of only semisimple elements may not be a semisimple Lie algebra.
\end{insight}

\begin{proposition}[Jordon-Chevalley Decomposition]
    Let $V$ be a finite dimensional vector space over $F$ and let $x\in \End V$. 
\begin{enumerate}[(a)]
    \item There exists a unique $x_s,\, x_n\in \End V$ satisfying the conditions: $x= x_s + x_n$, $x_s$ is semisimple, $x_n$ is nilpotent and $[x_n,x_s]=0$.
    \item There exists polynomials $p(T),\, q(T)$ in one indeterminate, without constant term, such that $x_s = p(x)$,\, $x_n = q(x)$. In particular, $x_n$ and $x_s$ commute with any endomorphism commuting with $x$.
    \item If $A\subset B \subset V$ are subspaces, and $x$ maps $B$ into $A$, then $x_s$ and $x_n$ also map $B$ into A.
\end{enumerate}
\end{proposition}

\subsection{Preservation of Jordon Decomposition}
\label{sub:preservation_of_jordon_decomposition}
\emph{Reference \cite{humphreys1972introduction} 6.4}

\begin{theorem}
    \label{thm:preservation_of_jordon_decomposition}
   Let $\LieG\in \LieGL(V)$  be a semisimple Lie algebra. Then $\LieG$ contains the semisimple and nilpotent parts in $\LieGL(V)$ of all of its elements. In  particular the abstract and usual Jordon decompositions coincide.
\end{theorem}

\begin{corollary}
    \label{cor:jordon}
    Let $\LieG$ be a semisimple Lie algebra. $\rho: \LieG\to \LieGL(V)$ a finite dimensional representation of $\LieG$. If $x=s + n$ is the abstract Jordon decomposition of $x\in L$, then $\rho(x)=\rho(s) + \rho(n)$ is the usual Jordon decomposition of $\rho(x)$.
\end{corollary}


