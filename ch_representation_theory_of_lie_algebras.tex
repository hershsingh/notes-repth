\chapter{Representation theory of Lie Algebras}
\label{cha:representation_theory_of_lie_algebras}



\section{Overview}
In this chapter we'll develop the full machinery that will allow us the completely classfify the semisimple finite Lie algebras over any algebraically closed field of characteristic zero.

\begin{enumerate}[(i)]
    \makethislistcompact
    \item Irreps of $\SL_2$
    \item General semisimple case
    \item Root space decomposition
    \item Cartan subalgebra
\end{enumerate}


\section{Irreps of \texorpdfstring{$\mathfrak{sl}_2$}{sl\_2}}
\label{sec:irreps_of_sl_2}

Let $\LieG=\LieSL(2,\FF)$ and $V$ be an arbitrary $\LieG$-module.
%Let $V$ be a $\SL_2$ module.

Choose
\begin{align}
    X = \begin{pmatrix} 1 & 0 \\ 0 & 0 \end{pmatrix}, \quad
    Y = \begin{pmatrix} 0 & 0 \\ 0 & 1 \end{pmatrix}, \quad
    H = \begin{pmatrix} 1 & 0 \\ 0 & -1 \end{pmatrix}.
\end{align}
Computing the commutators, we get
\begin{align}
    [H,X] = 2X,\quad [H,Y]=-2Y,\quad [X,Y]=H.
\end{align}
\Todo{$H$ acts diagonally on $V$, by corollary~\ref{cor:jordon}. Since $H$ is diagonalizable iff the eigenspaces span the vector space...}

We have the eigenspaces
\begin{align}
    V_\lambda = \{v\in V\ |\ H v = \lambda(v) v\}
\end{align}
where $\lambda(v)\in \FF$. This yields the decomposition
\begin{align}
    V &= \bigoplus_{\lambda\in \LieH^*} 
\end{align}
This yields a decomposition of $V$ as direct sum of eigenspaces 

When $V_\lambda \neq \{0\}$, we call $\lambda$ a weight of $H$ in $V$ and we call $V_\lambda$ a weight space.

\begin{lemma}
    If $v\in V_\lambda$ then $Xv \in V_{\lambda + 2}$ and $Yv\in V_{\lambda-2}$.
\end{lemma}
\begin{proof} Trivial.
\end{proof}
Finite dimensionality of $V$ implies the existence of a maximal $\lambda$. For such $\lambda$, any nonzero vector in $V_\lambda$ is called a \emph{maximal vector of weight $\lambda$}.

\subsection{Classification of irreducible modules}
\label{sub:classification_of_irreducible_modules}

Assume now that $V$ is an irreducible $\LieSL_2$ module. Choose a maximal vector $v_0\in V_\lambda$ and set
\begin{align}
    v_{-1}&=0, \\
    v_k &= \frac{1}{k!} Y^k v_0, \quad\text{for } k\geq 0.
\end{align}

\Todo{more stuff here...}

\section{Root Space Decomposition}
\label{sec:root_space_decomposition}

Let $\LieG$ be a semisimple Lie algebra.

\begin{definition}
    A Lie subalgebra $\Lie t \subset \Lie g$ is \emph{toral} if it contains only semisimple elements.
\end{definition}

\begin{insight}
    Where is the torus in the toral Lie algebra? 
\end{insight}

\begin{lemma} 
    Toral subalgebras exist.
\end{lemma}
\begin{proof}
    Every element $x\in \LieG$ can be decomposed as into nilpotent and semisimple parts $x=x_s + x_n$, which are also in $\LieG$. 
    If there was no semisimple element in $\LieG$, that is, if all the elements were nilpotent then $\LieG$ would be a nilpotent Lie algebra (by Engel's theorem). A semisimple Lie algebra can not be nilpotent (since nilpotent implies solvable). So, there exists some subalgebra (span of the semisimple elements) which is toral.
\end{proof}

\begin{lemma}
    A toral subalgebra is abelian.
    \label{lemma:toralsubalgebra_abelian}
\end{lemma}
\begin{proof}
    Let $\Lie t \subset \Lie g$ be a toral subalgebra. We need to show that $\ad_{\Lie t} x=0 $ for all $x\in \Lie t$. Since $x\in \Lie t$ is semisimple, $\ad_\Lie t x$ is semisimple and thus diagonalizable ($\FF$ is an algebraically closed field). If $\ad_\Lie t x \neq 0$, then it has linearly independent eigenvectors that span $\Lie t$. We need to show that all the eigenvalues are zero. Assume, on the contrary, that there is some non-zero $v\in \Lie t$ such that $\ad_\Lie t x (v)=av$ with $a\neq 0$. Now, notice that 
    \begin{align}
        \ad_\Lie t v(\ad_\Lie t v(x))=-a[y,y]=0
    \end{align} 
    and so $\ad_\Lie t v(x)$ is an eigenvector of $\ad_\Lie t v$ with eigenvector $0$. Since $v$ is semisimple,so is $\ad_\Lie t v$ and thus has eigenvectors that span the space. We can write $x$ as a linear combination of eigenvectors of $v$. Let $x=\sum_i a_i v_i$ where $\ad_\Lie t v (v_i)=\lambda_i v_i$ for $i=1,\dotsc,n$. Now
    \begin{align}
        \ad_\Lie t (\ad_t v(x))&= \ad_\Lie t (\sum_i a_i \lambda_i v_i) \\
            &= \sum_i a_i \lambda_i^2 v_i. 
    \end{align}
    But this must be zero, which contradicts the linear independence of the eigenvectors.  
\end{proof}

\begin{insight}
    What is the difference between Cartan subalgebras and maximal toral subalgebras?  \Todo{``Cartan subalgebra = maximal toral subalgebra'' is the correct definition only when $\Lie g$ is reductive.}
\end{insight}

\begin{definition}[Cartan subalgebra]
    The maximal toral subalgebra  is called the Cartan subalgebra.
\end{definition}

\begin{insight}
   For $\Lie {sl}(n,\FF)$, the maximal toral subalgebra is the set of all diagonal matrices.
\end{insight}
Let $\LieH$ be the Cartan subalgebra of $\LieG$. Since $\LieH$ is abelian (by lemma~\ref{lemma:toralsubalgebra_abelian}), 
\begin{align}
    [\ad_\Lie g H_1, \ad_\Lie g H_2] &= \ad_\Lie g [H_1, H_2] =0  \quad \forall H_1,\, H_2 \in \LieH.
\end{align}
So $\ad_\Lie g \Lie h$ is a commuting family of endomorphisms of $\LieG$. Therefore, $\ad_\LieG\LieH$ can be simultaneously diagonalized. In other words,
\begin{align}
    \LieG &= \bigoplus_{\alpha\in \LieH^*} \LieG_\alpha \\
    \text{with}\quad \LieG_\alpha &= \{x\in \LieG\ |\ \ad_\LieG h (x)=\alpha(h)x \quad\forall h\in \LieH\}.
\end{align}

Notice that $\LieG_0 = \{x\in\LieG\ |\ [\LieH,x] = 0 \forall x\in\LieG\} = C_\LieG(\LieH)$, the \defn{centralizer} of $\LieH$. 
\begin{insight}
    The centralizer of a subset is the set of all elements in the group which commute with all the elements in the subset.
\end{insight}

The set of all nonzero $\alpha\in\LieH^*$ for which $\LieG_\alpha \neq 0$ are called \defn{roots} of $\LieG$ relative to $\LieH$. We thus have the \defn{root space decomposition} or \defn{Cartan decomposition}: 
\begin{align}
    (*): \LieG &= C_\LieG(\LieH) \oplus \coprod_{\alpha\in\Phi} \LieG_\alpha
\end{align}

