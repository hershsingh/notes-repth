\chapter{Campbell-Baker-Hausdorff}
\label{cha:campbell-baker-hausdorff}
\section{}
We describe the algebraic proof of the CBH formula. 

\begin{enumerate}
    \makethislistcompact
    \item Define universal enveloping algebra
    \item Define the tensor algebra
    \item Show that universal enveloping algebra and the tensor algebra are canonically isomorphic
    \item Poincaré-Birkhoff-Witt theorem
\end{enumerate}

\subsection{Universal enveloping algebra}
\label{sub:universal_enveloping_algebra}

A \emph{universal enveloping algebra} is, in some sense, the most general associative algebra which contains a given lie algebra. 

Let $U(L)$ be an associative algebra with unit and $L$ be a lie algebra. $U(L)$ is the universal enveloping algebra if we have a map $\epsilon: L\to U(L)$ such that
\begin{enumerate}
    \makethislistcompact
    \item $\epsilon$ is Lie algebra homomorphism. This means that it is linear and it preserves the Lie bracket:
        \begin{align}
            \epsilon[x,y] &= \epsilon(x)\epsilon(y) - \epsilon(y) \epsilon(x)
        \end{align}
    \item it satisfies a \emph{universal property}. If $A$ is any associative algebra with unit and $\alpha: L \to A $ is any Lie algebra then there exists a unique $\phi: U(L)\to A$ such that the following diagram commutes.
        \begin{center}
            \begin{tikzpicture}[commutative diagram]
                \node (L) {$L$};
                \node (UL) [below right of=L] {$U(L)$};
                \node (A) [below left of=L] {$A$};
                \draw [arrow] (L) --(UL) node [midway, auto=left] {$\epsilon$};
                \draw [arrow] (L) --(A) node [midway, auto=right] {$\alpha$};
                \draw [arrow, dashed] (UL) --(A) node [midway, auto=below] {$\exists!\ \phi$};
            \end{tikzpicture}
        \end{center}
\end{enumerate}
\begin{insight}
    Given $(\epsilon_1, U_1)$ and $(\epsilon_2,U_2)$ we can see get maps $\phi_1: U_1\to U_2$  and $\phi_2: U_2\to U_1$ such that 
    $\phi_1\cdot\phi_2$ is identity \Todo{on $\epsilon_1(L)$} and 
    $\phi_2\cdot\phi_1$ is identity \Todo{on $\epsilon_2(L)$}. 
    But $\epsilon(L)$ generates $U(L)$, and $\phi$ is a associative algebra homomorphism, so 
    \begin{align}
        \phi\left(\sum \epsilon(x_1)\cdots\epsilon(x_n)\right) &= \sum \phi(\epsilon(x_1))\cdots\phi(\epsilon(x_n))\\
            &= \sum \epsilon(x_1)\cdots\epsilon(x_n).
    \end{align}
    Thus the compositions are identity on the whole of $U_1$ and $U_2$.
    
    %\Todo{How can we say that their composition is identity on the whole of $U_1$ and $U_2$?}
\end{insight}
Universal enveloping algebras are unique up to an isomorphism due to this universal property.

\subsection{Tensor Algebra}
\label{sub:tensor_algebra}
$i:V\to T(V)$ such that for any map $\alpha: V\to A$, where $A$ is an associative algebra, there exists a map linear map $\psi: T(V)\to A$ such that the following diagram commutes



\begin{center}
    \begin{tikzpicture}[commutative diagram]
        \node (V) {$V$};
        \node (TV) [below right of=V] {$T(V)$};
        \node (A) [below left of=V] {$A$};
        \draw [arrow] (V) --(TV) node [midway, auto=left] {$i$};
        \draw [arrow] (V) --(A) node [midway, auto=right] {$\alpha$};
        \draw [arrow, dashed] (UL) --(A) node [midway, auto=below] {$\exists!\ \psi$};
    \end{tikzpicture}
\end{center}


\subsection{Construction}
\label{sub:construction}
Let $L$ be a Lie algebra and let $I$ be the two sided ideal generated by the elements $[x,y]-x\otimes y + y\otimes x$, then
\begin{align}
    U(L) &= T(L)/I
\end{align} 
is a universal enveloping algebra for $L$.

We need to show that this satisfies the universal property. Set $V=L$ in the diagram for tensor algebra $T(V)$ to get

\begin{center}
    \begin{tikzpicture}[commutative diagram]
        \node (L) {$L$};
        \node (TL) [below right of=L] {$T(L)$};
        \node (A) [below left of=L] {$A$};
        \draw [arrow] (L) --(TL) node [midway, auto=left] {$i$};
        \draw [arrow] (L) --(A) node [midway, auto=right] {$\alpha$};
        \draw [arrow, dashed] (UL) --(A) node [midway, auto=below] {$\exists!\ \psi$};
    \end{tikzpicture}
\end{center}
If, additionally, $\alpha$ is a Lie algebra homomorphism, then 
\begin{align}
    \alpha[x,y] &= \alpha(x)\alpha(y) -\alpha(y) \alpha(x)\quad\forall x,y\in L.
\end{align}
But since $\alpha=\psi\cdot i$
\begin{align}
    \psi([x,y] -x\otimes y - y\otimes x) =0
\end{align}
Therefore $I\subset \ker \psi$, and we get that there is a unique map $\overline\phi: T/I \to A$ such that the following diagram commutes

\begin{center}
    \begin{tikzpicture}[commutative diagram]
        \draw [->, opacity=0.3] (0,-0.7) arc (90:-250:5pt);
        \draw [->, opacity=0.3] (0,-1.9) arc (90:-250:5pt);

        \node (L) {$L$};
        \node (TL) [below right of=L] {$T(L)$};
        \node (A) [below left of=L] {$A$};
        \node (T/I) [below left of=TL] {$T/I$};
        \draw [arrow] (L) --(TL) node [midway, auto=left] {$i$};
        \draw [arrow] (L) --(A) node [midway, auto=right] {$\alpha$};
        \draw [arrow] (TL) --(A) node [midway, auto=below] {$\psi$};
        \draw [arrow] (TL) --(T/I) node [midway, auto=below] {$\phi$};
        \draw [arrow, dashed] (T/I) --(A) node [midway, auto=below] {$ \overline{\phi}$};
    \end{tikzpicture}
\end{center}
Thus,  we have a lie algebra $T/I$ and a map $\epsilon=\phi\cdot i: L\to T/I$ such that the universal property is satisfied. \qed



\subsection{Extension of Lie algebra homomorphism to its UEA}
\label{sub:extension_of_lie_algebra_homorphism_to_its_uea}
Given map $h: L\to M$ between two lie algebras $L$ and $M$, the universal property implies the existence of a associative algebra homomorphism $U(L)\to U(M)$ as shown in the diagram:
\begin{center}
    \begin{tikzpicture}[commutative diagram]
        \node (L) {$L$};
        \node (M) [right of=L] {$M$};
        \node (UL) [below of=L] {$U(L)$};
        \node (UM) [right of=UL] {$U(M)$};
        \draw [arrow] (L) -- (M) node [midway, above] {$h$};
        \draw [arrow] (L) -- (UL) node [midway, left] {$\epsilon_L$};
        \draw [arrow] (M) -- (UM) node [midway, right] {$\epsilon_M$};
        \draw [arrow, opacity=0.3] (L) -- (UM) node [midway, auto,  shift={(-5pt,0pt)}] {$\epsilon_M\cdot h$};
        \draw [arrow, dashed] (UL) -- (UM) node [midway, below] {$\exists !$};
    \end{tikzpicture}
\end{center}


\subsection{UEA of a direct sum}
\label{sub:uea_of_a_direct_sum}

\begin{align}
    U(L_1\oplus L_2) &\isomorphic U(L_1) \otimes U(L_2)
\end{align}

\subsection{Bialgebra structure}
\label{sub:bialgebra_structure}
\begin{definition}[Bialgebra]
A vector space $C$ with a map (\emph{comultiplication}) $\Delta: C\to C\otimes C$ and a map (\emph{co-unit}) $\varepsilon: D\to k$ satisfying
\begin{align}
    (\varepsilon\otimes \id) \circ \Delta &= \id \quad\text{and} \\
    (\id\otimes \varepsilon) \circ \Delta &= \id
\end{align}
is called a \emph{co-algebra}. If $C$ is an algebra and both $\Delta$ and $\varepsilon$ are algebra homomorphisms, we say that $C$ is a \emph{bi-algebra}.
    
\end{definition}

Let $L$ be any lie algebra. The map $f: L\to U(L)\otimes U(L)$ defined by
\begin{align}
    x\mapsto f(x)=x\otimes 1 + 1\otimes x
\end{align}
is a Lie algebra homomorphism. This can be seen by
\begin{align}
    f(x)f(y)- f(y)f(x) &= [x,y]\otimes 1 + 1\otimes [x,y]\\
        &= f[x,y].
    %f([x,y]) &= f(x\otimes y - y\otimes x) 
\end{align}
Thus this map induces a map $\Delta: U(L)\to U(L)\otimes U(L)$ as seen in the following commutative diagram.
\begin{center}
\begin{tikzpicture}[commutative diagram]
    \node (L) {$L$};
    \node (ULxUL) [below right of=L] {$U(L)\otimes U(L)$};
    \node (UL) [below left of=L] {$U(L)$};
    \draw [arrow] (L) -- (ULxUL) node [midway, auto=left] {$f$};
    \draw [arrow] (L) -- (UL) node [midway, auto=right] {$\epsilon$};
    \draw [arrow, dashed] (UL) -- (ULxUL) node [midway, below] {$\exists! \Delta$};
\end{tikzpicture}
\end{center}
Now, define the map $\varepsilon: U(L)\to k$ as
\begin{equation}
    \epsilon(x) = \begin{cases} 
        1 &\text{if } x=1,\\
        0 &\text{for all } x\in L
    \end{cases}
\end{equation}
and extend this as an algebra homomorphism.
\begin{proposition}
   $(U(L), \Delta, \varepsilon)$  is a bialgebra.
\end{proposition}
\begin{proof}
    \Todo{complete this} 
\end{proof}

\subsection{The Poincar\'e-Birkhoff-Witt Theorem.}
\label{sub:the_poincar'e_birkhoff_witt_theorem_}

\Todo{Complete this section.}

\begin{theorem}[Poincar\'e-Birkhoff-Witt]
    \begin{align}
        S(L) \isomorphic \gr U(L)
    \end{align}
\end{theorem}




