\chapter{Induced Representations}
\label{cha:induced_reprentations}

This chapter is based heavily on \cite{etingof_introduction_2009}.
\section{Overview}
\begin{enumerate}
    \makethislistcompact
    \item Define induced representations
    \item Mackay formula for the character of an induced representation.
    \item Frobenius Reciprocity
\end{enumerate}

\section{Definitions}

\Todo{Don't really understand the motivation for this section.}

Given a subgroup $H\subset G$, and $(\rho_V, V)$, a representation of $G$, we can easily construct a representation for $H$ by simply restricting $\rho_V$ to $H$, which we denote by $\rho_V|_H$.

However, if want to construct a representation of $G$ from a representation of $H$, then we need to think a little more about it. 

Think of the elements of $V$ as functions of $G$, that is, every element $v\in V$ defines a map $v: G\to V$ given by
\begin{align}
    v(g) &= \rho(g) v.
\end{align}
Now, we already know how $v(\cdot)$ acts on $H$. We want to extend that action to $G$. The least we could do is to ensure that the action on $G$ is consistent with the known action on $H$. So, we enforce that
\begin{align}
    v(hg) = h\cdot v(g) &= \rho_V(h) v(g)\quad \forall h\in H,\ g\in G.
\end{align}
The trick is to take the space of \emph{all} such functions $v: G\to V$ that satisfy the above condition. If that turns out to be a representation, we would get the most \emph{free} representation. So, lets define 
\begin{align}
    \Ind_G^H(V) &= \{f: G\to V : v(gh)=\rho_V(h)v(g)\ \forall h\in H, g\in G\}.
\end{align}

\Todo{
   Check this is indeed a representation. 
   }


\pagebreak
\begin{insight}
    Why is $\Ind_H^G{V}$ naturally isomorphic to $\Hom_H(k[G],V)$?
    An element of $\Hom_H(k[G],V)$ is just a $H$-module homomorphism $f: k[G]\to V$ such that
    \begin{align}
        f(hg)  = h\cdot f(g)
        \implies f(hg) &= \rho(v) f(g)
    \end{align}
    All such $f$ are precisely all the elements in $\Ind_H^G(V)$.
\end{insight}
