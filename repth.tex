\documentclass[a4paper]{book}
\usepackage{amsmath,amssymb}
\usepackage{hyperref}
\usepackage{amscd}
\usepackage{color}

% Framed environments
\usepackage[framemethod=tikz]{mdframed}

% Preamble
\DeclareMathOperator\GL{GL}
\DeclareMathOperator{\Image}{Im}
\DeclareMathOperator{\Coker}{Coker}
\DeclareMathOperator{\Ker}{Ker}
\DeclareMathOperator{\Hom}{Hom}
\DeclareMathOperator{\Sym}{Sym}
\newcommand{\CC}{\mathbb{C}}
\newcommand{\RR}{\mathbb{R}}

\renewcommand{\>}{\rangle}
\newcommand{\<}{\langle}

% Environment for "physical insights"
\newenvironment{insight}
  {\begin{mdframed}[%style=2,%
      leftline=true,
      rightline=true,
      topline=false,
      bottomline=false,
      leftmargin=10em,
      rightmargin=2em,%
      innerleftmargin=1em,
      innerrightmargin=1em,
      linewidth=2pt,%
      linecolor=white!70!black,%
      %backgroundcolor=white!99!black, %
      skipabove=7pt,skipbelow=7pt]\small}
  {\end{mdframed}}

%%%%%%%%%%%%%%%%%%%%%%%%%%%%%%%%

% Title
\title{Representation Theory}
\author{Hersh Singh}
\begin{document}
\maketitle
\chapter{Representations of Finite Groups}
\section{Definitions}
\label{sec:definitions}

Let $G$ be a finite group, $V$ be a finite dimensional complex vector space.

A \emph{representation} of $G$ on $V$ is a group homomorphism 
\begin{align}
    \rho:  G\to \GL(V), 
\end{align}
where $\GL(V)$ is the group of automorphisms of $V$. Even though the ``representation" is really the homomorphism, but it is common (especially by physicists) to refer to $V$ itself as the representation. 

\begin{insight}
We say that this map gives $V$ the structure of a G-module. This agrees with the  definition of a R-module (R is a ring) that I have studied earlier. An R-module is simply a vector space defined over a ring, instead of a field. So the ring elements act as the \emph{scalars}. Of course, the description must include a rule for the `interaction' of the scalars with the vectors. To be a module, the interaction must be \emph{linear}.  
In this case, the group homomorphism $\rho$ gives us that rule of interaction between the vectors (elements of $V$) and the scalars (elements of $G$). For $g\in G, v \in V$, $$gv\equiv\rho(g)v \in V$$
\end{insight}

A vector space homomorphism $\phi: V\to W$ is a morphism between the two representations $V$ and $W$ if the following diagram commutes:
\begin{equation} \begin{CD}
  V @>\phi>> W\\
  @VgVV @VVgV\\
  V @>\phi>> W\\
\end{CD} \end{equation}

That is, $\phi g=g\phi$ for all $g\in G$. This makes the group elements \emph{behave like scalars under module homomorphisms}. Such morphisms of representations are also called $G$-\emph{linear map} or a \emph{$G$ intertwiner}.

\begin{insight}
   Why is this is a good definition? Seems to be inspired from module homomorphisms. This is natural in some sense - Figure that out.
\end{insight}

$\Ker \phi,\ \Image \phi,\ \Coker \phi = V/\Image\phi$ are also $G$-modules. This is solely because of the commutativity of the above diagram.
\begin{itemize}
    \item If $v\in \Ker \phi$, then $\phi(gv) = g\phi(v) = 0$. So $gv$ also $\in \Ker \phi$. $\blacksquare$.
    \item If $v\in \Image \phi$, then let $\phi(v) = w$. So $\phi(gv)=g\phi(v)=gw \in W$. So $gv$ also $\in \Image \phi$. $\blacksquare$.
\end{itemize}

One of the goals of our study is, given a representation, to develop tools for constructing other, preferably all, representations of the group. 
Some examples of representations that can be constructed from $V$ and $W$

\begin{itemize}
    \item Tensor product $V\otimes W$ via
        \begin{align}
            g(v\otimes w) = g(v) \otimes g(w)
        \end{align}
    \item Tensor power $V^{\otimes n}$ and the \emph{exterior power}  $\Lambda^n(V)$ and the \emph{symmetric power} $\Sym^n(V)$ are its subrepresentations.
    \item the dual $V^* = \Hom(V,\CC)$. This is a little tricky though. The action of $G$ on $V^*$ must be such that it preserves the natural inner product, denoted by $\<\ ,\ \>$, between them.  This forces us to define the action of $g$ such that 
        \begin{align*}
            \rho^*(v^*) &=
        \end{align*}
\end{itemize}

\chapter{General Information}


\section{Resources}
\label{sec:Resources}
Here are some resources that I found useful while preparing these notes
\begin{itemize}
    \item Representation Theory - A First Course by \emph{Fulton, Harris}
\end{itemize}







\end{document}
