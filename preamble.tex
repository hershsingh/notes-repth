%\RequirePackage[l2tabu,orthodox]{nag}

\documentclass[oneside]{book} 
% the 'oneside' option makes viewing on a computer easier, as both inner and outer margins are equal. Change this while printing.

\usepackage{amsmath,amssymb}
\usepackage[linkbordercolor={0.9 0.9 1}]{hyperref}
\usepackage{amscd}
\usepackage{amsthm}
\usepackage{color}

\usepackage{microtype}

\usepackage[T1]{fontenc}
\usepackage{lmodern}

% Inconsolate as the monospace font
%\usepackage{inconsolata}

\usepackage{enumerate} % Customize the enumerate environment
\usepackage{mathtools} % for \prescript

%\usepackage{cite}
\usepackage{bm} 

\usepackage[vcentermath]{youngtab} % Excellent package for young diagrams

% For compact lists. Provides the environments 
%   itemize*, enumerate*, description*
% with lesser spacing between items.
\usepackage{mdwlist} 
\usepackage{enumerate} 

\usepackage{mathtools} % For 'Aboxed' command

% Theorem-like environments
\usepackage{amsthm}
\theoremstyle{definition}
\newtheorem{definition}{Definition}

\theoremstyle{plain}
\newtheorem{theorem}{Theorem}
\newtheorem{corollary}{Corollary}
\newtheorem{conjecture}{Conjecture}
\newtheorem{lemma}{Lemma}
\newtheorem{proposition}{Proposition}

% TikZ
\usepackage{tikz}
\usepackage{tikz-cd} % Commutative Diagrams in TikZ
\usetikzlibrary{positioning}
\usetikzlibrary{decorations.pathreplacing} % for curly braces
\usetikzlibrary{decorations.pathmorphing} % for squiggly arrows
\usetikzlibrary{calc} % for coordinate calculation
% Tikz Style for Commutative Diagrams
\tikzset{
    commutative diagram/.style={
            node distance=2cm,auto,
            arrow/.style={-stealth},
            exists/.style={-stealth,densely dotted}
        }
    }

% This compactifies a list by removing unnecessary whitespace
\newcommand\makethislistcompact{
        \setlength{\itemsep}{0pt}%
        \setlength{\parskip}{0pt}%
        %\setlength{\topsep}{50pt} 
        %\setlength{\partopsep}{10pt}
        %\setlength{\parsep}{50pt}
}
% Framed environments
\usepackage[framemethod=tikz]{mdframed}

% Common operators
\DeclareMathOperator\GL{GL}  % General Linear Group
\DeclareMathOperator{\Image}{Im} % Image
\DeclareMathOperator{\Coker}{Coker} % Cokernel 
\DeclareMathOperator{\Ker}{Ker} % Kernel
\DeclareMathOperator{\Tr}{Tr} % Trace
\DeclareMathOperator{\Hom}{Hom} % Homomorphisms
\DeclareMathOperator{\End}{End} % Endomorphisms
\DeclareMathOperator{\Aut}{Aut} % Automorphisms
% Commonly used sets
\newcommand{\CC}{\mathbb{C}}
\newcommand{\RR}{\mathbb{R}}
\newcommand{\ZZ}{\mathbb{Z}}
\newcommand{\HH}{\mathbb{H}}
\newcommand{\FF}{\mathbb{F}}
\newcommand{\uhp}{\mathcal{H}}
\newcommand{\SL}{\text{SL}}
\newcommand{\SU}{\text{SU}}

\renewcommand{\>}{\rangle}
\newcommand{\<}{\langle}

% Some semantically named symbols
\newcommand\isomorphic\cong

% Meta-notes
\newcommand\Solution[2]{\emph{[Solution to Exercise #1 in #2]}}
\newcommand\Todo[1]{\begin{color}{red}{#1}\end{color}}
\newcommand{\tocheck}[1]{{\color{red} #1}}
\newcommand{\comment}[1]{{\color{cyan} #1}}

% Environment for "physical insights"
\newenvironment{insight}
  {\begin{mdframed}[%style=2,%
      leftline=true,
      rightline=true,
      topline=false,
      bottomline=false,
      leftmargin=2em,
      rightmargin=2em,%
      innerleftmargin=1em,
      innerrightmargin=1em,
      linewidth=2pt,%
      linecolor=white!70!black,%
      %backgroundcolor=white!99!black, %
      skipabove=7pt,skipbelow=7pt]\small}
  {\end{mdframed}}

%%%%%%%%%%%%%%%%%%%%%%%%%%%%%%%%

